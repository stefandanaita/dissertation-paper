\chapter{Conclusions}

In this report we have outlined the design and implementation of a TV Series tracking system. We proposed this project since, by using the available resources existent at this hour on the market, we came across flaws, inefficiencies or features that could be improved. 

After carefully reviewing all the resources available, we decided upon the mandatory or optional features we desired our system to have, as presented in chapter 3. The system has been designed so that it accomplished as many features as possible.
This report's focus is on the back end part of the system - the web server, database and the recommender, while my teammate focuses on the actual design of the website and of the mobile application. 

Another vital part was the recommendation system. After studying a history of these software products, which started only as recommenders for TV programmes, then extended to movies and series, we chose to use a method based on collaborative filtering, as presented in chapters three and four. This is the known method used by Netflix at the present, after organising a competition in which many programmers around the world exposed their research and ideas. We have implemented this part using the Likely npm package, which, at a more basic level, uses the methods highlighted in the Netflix system.

The result of the evaluations have been rather impressive as the system scaled up on the benchmarks tests, apart from the TV shows import approach which is not working properly because of TheMovieDB's API requests limit. 

Nevertheless, there is room for improvement left and chapter six highlights the limitations of the current system as well as methods for improvement. We want to continue the development of the project and put it on the market.